\documentclass[11pt]{book}
\usepackage{graphicx}
\usepackage[margin=1in,bindingoffset=.2in]{geometry}
\usepackage[backend=biber, style=ieee]{biblatex}
\addbibresource{references.bib}
\usepackage[parfill]{parskip}

\begin{document}
\begin{titlepage}
	\newgeometry{margin=1in}
	\begin{center}
		{\huge Advanced Querying and Analysis\\ of Git Repositories\\}
		\vspace{1.5cm}
		{\Large \textbf{Daniel Brown} \\}
		{\Large Department of Computer Science, University of York \\}
		\vspace{1.5cm}
		\includegraphics[width=100px]{images/university-of-york-shield} \\
		\vspace{1.5cm}
		{\Large September 2015 \\}
		\vspace{1.5cm}
		\Large Supervised by Dr. Dimitris Kolovos \\
		\vspace{1.5cm}
		\Large A thesis submitted in partial fulfilment for the degree of \\ \textit{Master of Science in Advanced Computer Science}\\
		\vspace{5cm}
		\small XX,XXX words as counted by the `detex thesis.tex | wc -d` unix commands
	\end{center}
	\restoregeometry
\end{titlepage}

\chapter*{\centering Abstract}
\addcontentsline{toc}{chapter}{Abstract}
\begin{center}
	\parbox{350pt}{
Git is one of the most popular version control systems available \cite{gitpopularity} and, as well as many privately hosted instances, powers the well-known social programming website GitHub \cite{gitpowersgithub}.

Despite Gits popularity there are few systems available for analysing a git repository or querying it for useful metrics such as most popular commit time, largest commit, most active contributor or something more complex.

The systems that do exist to analyse Git repositories, namely GitInspector \cite{gitinspector} and Github, don't allow for custom queries to be written by the user. Meaning, for example, you cannot ask the question "What day of the week is Daniel most likely to author a commit of over 100 lines" or "Which contributor has had the highest percentage of their committed lines changed by someone else". 

This project presents a tool, called EpsilonGit, which allow users to write these custom queries by letting them interact with a git object database as a model using the Epsilon platform and its associated domain specific modelling languages. This model-based solution is then compared with existing technologies on the parameters of code complexity, speed, and extensibility by the user.
}
\end{center}

\tableofcontents

\chapter{Introduction and Background}

\section{Acknowledgement}
I would like to take this opportunity to thank Dr. Dimitris Kolovos who not only provided support and encouragement throughout the course of this project, but also came up with the original idea of using model driven engineering to interact with git and added features into the Epsilon suite to support some of the work I was doing.

A special thank-you is extended to my parents, siblings and girlfriend who have supported me throughout the trials and tribulations of my time at York by visiting both myself and the National Railway Museum an inordinate amount of times. The same gratitude is required of my friends both from at home in Dunstable and The University of Hull, where I completed my Bachelors degree. %The museum bit is an inside joke. I'll keep it in for a laugh

\section{Statement of Ethics}
Ethics in research is of paramount importance and therefore this project has been carried out with them in mind at every step. 

All literature, ideas and code that has been used in the formation of this project has been correctly referenced and credit has been given where it is due.

The data that my project interacts with is a local copy of a git database. The user must source this data on their own, and therefore the project doesn't have to deal with any authentication or security. EpsilonGit only gives people a different view on data they already have.

No other significant ethical issues related to this project could be identified.

\clearpage

\section{Introduction}
% Rough outline of git 
	% (e.g. its a distributed VCS... so people have code locally to be modelled, people use it for x functions (e.g. commit, branch, put their name to code) )
	% Why would someone what to query and analyse their git repository?
	% Some cool questions that could be asked
	
Git is a free and open source distributed version control system \cite{gitintro} that is designed to allow users to manage changes to files, often source files in a software project. This enables developers to "roll back" to a previous version of a file, see the difference between a file at two different times or determine which team member authored a line of code. 

Version control best practices suggest that a developer should be committing code a little at a time and quite often \cite{gitbestpractices}, this makes it a good place to analyse to determine an individual developer or teams working practices in retrospect.

A project manager may want to analyse a git repository to answer questions such as "How large is the average commit size?", "What time of the day are most commits made?" and "Which of my developers is committing the most code that is later replaced?". The answers to these questions may allow improvements in both quality of code and workplace practices.

% Rough outline of modelling
	% What is modelling?
	% Why is it _possibly_ suitable for this project?
Model Driven Engineering is an approach to tackle the complexity of data, and how it is interacted with, through the use of high level abstractions called models \cite{modeldrivenengineering} and a set of Domain-specific modelling languages and Transformation engines and generators.   
	
Due to the large array of features provided by git and the widespread use of hashes and graph data structures in the underlying system it is often considered to be complex \cite{gitcomplex}\cite{githard}\cite{gitmixedmetaphors}.

In this paper it is proposed that a model driven engineering solution to querying and analysing git repositories is a good solution as it removes much of the complexity associated with git.

% Motivation
	% Improve companies, open source projects and individuals understanding of their code and workflows
	% Encourage people to use and invest in Modelling, particularly using epsilon
	% Release it to the world and see what innovative and cool information people could get from their git repositories. With open data and software all kinds of weird shit can be done the original creator wouldn't have thought of.
\subsection{Motivation}
The Joel Test \cite{joeltest} is often cited as a simple list of best practices for software development. Rule number 1 is to use version control software in order to aid teamwork, maintain a canonical history of source code and reduce the likelihood of losing any work. The positive impact that version control has means that many open source, individual and commercial projects use version control -- one of the most popular being git \cite{gitpopularity}. 

Although many projects use git there are very few tools for querying and analysing the metadata and other information contained within git repositories. Those that are available lack the ability to add user-designed custom queries. 

Through developing a system in which users can write their own custom queries in an easy-to-use and abstract fashion it is thought that teams may be able to learn more about their own workflows and where improvements can be made -- this could be particularly useful to teams using agile methodologies which lack the rigid structure of older methods such as Waterfall.

The software produced as part of this project should also make it easier to determine if git best practices \cite{gitbestpractices} are being followed. In addition to best practices some teams set project wide best practices for commit messages \cite{erlanggitcommitmessages}, users should be able to write validators to ensure these are being used.

As well as helping teams understand their git repositories this project also aims to encourage Software Engineers to learn about and invest in Model-Driven Engineering. It is the the hope of the author that combining MDE with a technology as prevalent as git may help to widen its appeal. In particular it would be great if this attracted more people to the open source Epsilon MDE platform \cite{epsilonhomepage}.

One of the %TODO:
	
% Aims and objectives
	% Develop a solution which covers all of the most common parts of git (e.g. Object Model, Branches, Authors and Committers)
	% Develop a solution which has as fast or faster runtime for the same output as GitInspector / Github / GitSQL
	% Develop a solution which requires less code, and code of lower complexity for the same output as other solutions
	% Develop a solution which, at the same times, provides high level simple nice clean interaction whilst allowing access to low level stuff for those who also want that interaction

% Report Structure
	% A paragraph or two explaining the overall flow of this document, and calling out the names of each chapter. Not repeating the table of contents, but instead highlighting important sections in prose.

\section{Background}
% More in-depth Explanation of Git Object Model
% More in-depth explanation of modelling
	% What is a model?
	% How do we interact with models?
	% etc?
% Discussion of epsilon
% Discussion of the epsilon driver framework and what integrating with epsilon gets us (e.g. ability to make HTML from models, use of EOL Language, ability to run in and out of eclipse, etc)

\chapter{Literature Review}
% Intro paragraph "This chapter looks at the existing literature... etc"
% Section on literature about Git and the git object model
	% Conclusions drawn from the literature
		% A well known model, with some interesting idiosyncacies. Fits modelling well.

% Section on literature pertaining to accessing git information
	% e.g. http://www.researchgate.net/publication/279058070_Gitana_a_SQL-based_Git_Repository_Inspector
	% Some info on gitinspector
	% Conclusions drawn from the literature
		% Primarily, there isnt many projects doing this. The SQL one has some drawbacks, etc.
	
% Section on literature pertaining to analysing and querying git. (there is very little if not none)
	% Conclusions drawn from the literature
		% primarily, its novel

% Section on interacting with object models, or version control software via models. Perhaps the HTML DOM? 

\chapter{Methodology}
% Used agile TDD based approach, discuss why this was chosen and

\chapter{Requirements}
% Statement of Needs
% Stakeholder Identification (whos interested in project and their roles)
	% requirements of these stakeholders
% Breakdown of functional (features) and non-functional (speed, UX) requirements
% Use case diagram

\chapter{Design}
% Making architecture fit between JGit and Epsilon
% Experimentation and changes necessary as things changed (for example learning more about jGit, epsilon and Java or subclassing jGit types to provide better names in EOL code and hide some of the underlying implementation)
% Language
	% Explain how expected EOL code was developed before driver was, so driver could achieve what was wanted  

\chapter{Implementation}
% Environment used
	% Epsilon Eclipse Interim (why interim?)
	% Latest version of Git
% Libraries used
	% JGit
	% Apache Commons Lang
	% etc.
% Projects
	% .git.dt (for creating new Models from git)
	% .git (main functionality of driver)
	% .git.test (tdd stuff)
	% .git.tools (some useful EOL tools for Git stuff)
% Lots of stuff about implementation decisions here
% Summary of what do to to get the system running on your own machine

\chapter{Testing}
% Explain use of TDD, both unit tests, integration and use of EOL files that it would be hoped would work with
% Automated testing
% Use of CI
% Testing against requirements
% Summary (did all test pass?)

\chapter{Evaluation}
% Does it at least fulfil all of the analysis and querying required by the two systems its competing with (github and gitinspector?)
% Case Study vs GitInspector
	% Comparison of lines of code and code complexity required for the same output
	% Comparison of Run Time on same hardware

\chapter{Conclusion}
% Project Review (Cronological rundown of what happened)
% Is this a feasible way of dealing with git analysis and querying?
	% Why? Why not?
% Whats missing?
% A success?


\section{Future Work}
% Currently read only, would be interesting to see if being able to author commits etc via epsilon could be useful. Or even the ability to change names on commits etc via http://stacktoheap.com/blog/2013/01/06/using-mailmap-to-fix-authors-list-in-git/
% Make it would with subversion, mercurial etc
% Transform from git to svn/mercurial or vice versa via ETL (lots of people would want to make this transition)
% Lots of low hanging fruit regarding performance (e.g. computing properties each time instead of storing them in memory once they've been worked out once)

\chapter{Reflection}
% Working on project whilst other 'real life' stuff happens
% Working on coming back to a project after a long time 
% TDD was great, agile working seemed to work well.
% Java code is pretty reasonable, took a little while to get into but previous C# knowledge helped
% Example EOL/EGX code works well, but might be better if more declarative 
% This is one of only a few large projects I've worked on, so it was good experience

\printbibliography

\end{document}
