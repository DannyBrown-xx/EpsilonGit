\documentclass[runningheads,a4paper]{llncs}
\usepackage{amssymb}
\setcounter{tocdepth}{3}
\usepackage{graphicx}
\usepackage{multirow}
\usepackage{booktabs}
\usepackage{algorithm2e}
\usepackage [english]{babel}
\usepackage [autostyle, english = american]{csquotes}
\MakeOuterQuote{"}

\renewcommand\dblfloatpagefraction{.99}
\renewcommand\dbltopfraction{.99}
\renewcommand\floatpagefraction{.99}
\renewcommand\topfraction{.99}
\renewcommand\bottomfraction{.99}
\renewcommand\textfraction{.01}

\usepackage{url, listings, color} 
\newcommand{\keywords}[1]{\par\addvspace\baselineskip
\noindent\keywordname\enspace\ignorespaces#1}

\definecolor{mygray}{rgb}{0.95,0.95,0.95}
\lstset{frame=none,
  backgroundcolor=\color{mygray},
  language=Octave,
  aboveskip=3mm,
  belowskip=3mm,
  showstringspaces=false,
  columns=flexible,
  basicstyle={\small\ttfamily},
  xleftmargin=15pt,
  numbers=left,
  breaklines=true,
  breakatwhitespace=false,
  tabsize=2,
}

\begin{document}

\title{A Comparative Analysis of Eclipse Modelling Git Repositories}
\author{Daniel Brown\and Dimitrios S. Kolovos}
\institute{Department of Computer Science, University of York, United Kingdom \\ \{dtb508,dimitris.kolovos\}@york.ac.uk}
\maketitle

\begin{abstract}
Git is one of the most popular version control systems available \cite{gitpopularity} and, as well as many privately hosted instances, powers the well-known social programming website GitHub \cite{gitpowersgithub}.

Despite Gits popularity there are few systems available for analysing a git repository or querying it for useful metrics such as most popular commit time, largest commit, most active contributor or something more complex.

The systems that do exist to analyse Git repositories, namely GitInspector \cite{gitinspector} and Github, don't allow for custom queries to be written by the user. Meaning, for example, you cannot ask the question "What day of the week is Daniel most likely to author a commit of over 100 lines" or "Which contributor has had the highest percentage of their committed lines changed by someone else". 

This workshop presents a tool, called EpsilonGit, which allow users to write these custom queries by letting them interact with a git object database as a model using the Epsilon platform and its associated domain specific modelling languages. This model-based solution is then used to make some basic queries against some Eclipse Modelling Git repositories.
\end{abstract}

\section{Introduction}


\section{Background}

\section{Querying Git Repositories with Epsilon}

\section{Analysis of Eclipse Modelling Git Repositories}
Compute and present statistics related to the size/activity in the Git repositories of the main Eclipse modelling repositories:

\begin{itemize}
  \item git://git.eclipse.org/gitroot/m2t/org.eclipse.jet.git (JET2)
  \item git://git.eclipse.org/gitroot/m2t/org.eclipse.acceleo.git (Acceleo)
  \item https://github.com/eclipse/xtext.git (Xtext)
  \item git://git.eclipse.org/gitroot/epsilon/org.eclipse.epsilon.git (Epsilon)
  \item git://git.eclipse.org/gitroot/mmt/org.eclipse.atl.git (ATL)
  \item https://git.eclipse.org/r/sirius/org.eclipse.sirius (Sirius)
  \item git://git.eclipse.org/gitroot/mmt/org.eclipse.qvto.git (QVTo)
  \item https://git.eclipse.org/r/emf/org.eclipse.emf (EMF)
  \item https://git.eclipse.org/r/incquery/org.eclipse.incquery (IncQuery)
  \item https://git.eclipse.org/r/gmf-tooling/org.eclipse.gmf-tooling (GMF Tooling)
  \item git://git.eclipse.org/gitroot/gmf-runtime/org.eclipse.gmf-runtime.git (GMF Runtime)
  \item git://git.eclipse.org/gitroot/gmf-notation/org.eclipse.gmf.notation.git  (GMF Notation)
  \item https://git.eclipse.org/r/ocl/org.eclipse.ocl (OCL)
  \item https://git.eclipse.org/r/henshin/org.eclipse.emft.henshin (Henshin)
  \item git://git.eclipse.org/gitroot/m2t/org.eclipse.xpand.git (Xpand)
  \item https://git.eclipse.org/r/papyrus/org.eclipse.papyrus (Papyrus)
  \item git://git.eclipse.org/gitroot/uml2/org.eclipse.uml2.git (UML2)
\end{itemize}

Possible statistics (can be computed for the lifetime of the repositories or for the last 3 years)
\begin{itemize}
  \item Total lines of code
  \item Lines of code per file extension 
  \item Number of active committers (e.g. with commits in the last 30 days)
  \item Median number of files/lines changed per commit (smaller commits are generally better)
  \item \ldots
\end{itemize}

\section{Conclusions and Further Work}


\bibliographystyle{splncs03}
\bibliography{references}

\end{document}
